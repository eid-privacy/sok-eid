\section{Introduction}

With the large-scale deployment of state-backed digital identity systems fast approaching \cite{Swiyu} \cite{EUDI-ARF}, we investigate how holders' privacy can be improved in the proposed technological stack of \swiyu.
\swiyu is the Swiss proposal for a digital identity card.
The law proposal and its implementation are up for popular vote by the end of September 2025.
This partial and preliminary paper is the first milestone of our investigation. 
It is the first part of a \emph{Systematization of Knowledge} paper defining terminology, and pointing to the technological state of the art useful when discussing digital identity systems.
In the upcoming second part, we will add evaluation criteria to analyze, evaluate, and compare digital identity systems.
In particular, we focus on what could improve or detract from individual's privacy guarantees when using an \eid system.

\section{Document Structure}

The focus of this document is a taxonomy of digital identity systems that maps out and defines the core concepts of such systems.
We start by defining the actors in such a system \ref{subsub:actors}, followed by their objectives and expectations when participating \ref{subsub:trust_goals}, and end with a description of the threats to such a system \ref{subsub:threat_models}.

Then we move on to outline the state-of-the-art in cryptography and security as it relates to identity systems \ref{sub:foundations}: signature schemes, zero-knowledge protocols, and secure hardware.

Finally, we review two legal frameworks, EUDI \cite{EUDI} and the Swiss e-ID act \cite{SwiyuAct25}, that regulate the design, implementation, and use of these systems \ref{sub:regulatory}.

We finish this preliminary paper with a more detailed outlook of the second part in \ref{sec:follow-up}.