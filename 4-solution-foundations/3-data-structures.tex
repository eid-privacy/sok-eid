\subsubsection{Data Structures}

The following data structures are used in the EUDI and \swiyu projects.

\begin{itemize}
    \item Verifiable Credentials (VC) is a generic term to define a data structure containing attributes from the holder. 
    They are signed by an issuer, so that verifiers can check the validity of the credential.
    \item mDoc / mDL a type of VC defined by the ISO standard \cite{ISO18013-5_2021}.
    The VC only includes salted hashes of the attributes, which allows for selective disclosure.
    It is used as the VCs in the EUDI, and rolled out as the driving license in some states of the USA.
    \item SD-JWT is an open standard VC hosted by the IETF \cite{SDJWT}.
    The VC only includes salted hashes of the attributes, which allows for selective disclosure.
    It is used as VC in EUDI and \swiyu.
\end{itemize}

% \dirtree{%
% .1 Data Structures. 
% .2 Credential Formats. 
% .3 IETF SD-JWT. 
% .3 ISO mDoc. 
% .3 W3C Verifiable Credentials. 
% .3 AnonCreds. 
% .3 AnonCreds v2. 
% .2 Trust Lists. 
% .3 EBSI Trust List. 
% .3 ETSI Trusted Lists. 
% .3 Swiyu Trust Protocol. 
% .2 Certificates. 
% .3 X.509. 
% .2 Pseudonymity Schemes.
% .3 W3C DIDs. 
% .2 Revocation Lists. 
% .3 W3C Status Lists. 
% .3 IETF Status Lists. 
% .3 Anoncred's accumulators. 
% }
