\subsection{\rot}

The \rot defines the basis of what we want to achieve: who trusts whom, and what can go wrong (threats).
This is crucial for the analysis of the inevitable trade-offs that comprise such a large solution:

\begin{itemize}
    \item is it bad if a [given actor] can do this [given action]?
    \item what can go wrong, and how likely is it to go wrong?
    \item can we ignore certain threats because they are unlikely enough?
\end{itemize}

We start by giving a list of \emph{actors} with their name and a definition of their expectations.
Then we define the \emph{trust goals} that we want to achieve in order to create a useful service.
Finally we go over the \emph{threat models} we consider.
