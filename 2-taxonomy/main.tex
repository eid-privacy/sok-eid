\section{Taxonomy}
\label{sec:taxonomy}

We based our taxonomy on a similar work \emph{Privacy-enhancing technologies for digital payments: mapping the landscape} \cite{ABCD25} and discussions
in 2024 / 2025 around the electronic identity proposals of the EU \cite{EUDI-ARF}
and Switzerland \cite{Swiyu}.
The first level of taxonomy separates three domains of concern
regarding privacy for digital identity users. We further split each domain in
subgroups:

\begin{forest}
for tree={
    grow=0,                    % Left to right
    parent anchor=east,        % Connect from right side of parent
    child anchor=west,         % Connect to left side of child
    l sep=2cm,                % Horizontal separation
    %s sep=1cm,                % Vertical separation between siblings
    text width=4cm,
    align=left,
    reversed
}
[Privacy-Preserving\\Electronic Identities
    [\rot
        [Actors]
        [Trust Goals]
        [Threat Models]
    ]
    [Solution Foundations\\and Implementations
        [Cryptographic\\Building Blocks]
        [Hardware\\Security]
%        [Data\\Structures]
%        [Protocols]
    ]
    [Regulatory and\\Normative Frameworks
        [Legal Texts]
%        [Technical\\References]
%        [Standards]
    ]
]
\end{forest}

In \emph{\rot}, we list the goals and incentives for each party in an identity system. It is a list of concepts that an \eid must address in order to offer a trustworthy solution to all parties.
It is followed by the threats a good solution must mitigate in order to be trustworthy.
\emph{\sfi} lists technical means to achieve these goals and mitigate risks.
The \emph{\rnf} branch presents the legal texts from the EU and Switzerland related to \eid that constrain the solutions and technology that can be deployed, limiting the scope of what we consider in this paper.

\subsection{\rot}

The \rot defines the basis of what we want to achieve: who trusts whom, and what can go wrong (threats).
This is crucial for the analysis of the inevitable trade-offs that comprise such a large solution:

\begin{itemize}
    \item is it bad if a [given actor] can do this [given action]?
    \item what can go wrong, and how likely is it to go wrong?
    \item can we ignore certain threats because they are unlikely enough?
\end{itemize}

We start by giving a list of \emph{actors} with their name and a definition of their expectations.
Then we define the \emph{trust goals} that we want to achieve in order to create a useful service.
Finally we go over the \emph{threat models} we consider.

\subsubsection{Actors}
\label{subsub:actors}

In \emph{\rot}, \emph{Actors} describes the parties interacting through the identity system from two different angles: their \emph{legal status} and their \emph{role} in an interaction.
Depending on the type of an actor and their functional role, the privacy requirements and threat models differ.
The legal entities are based on the definition in most of the legal systems: private individuals, businesses, and governments.
We elected to use the functional roles that are commonly used in the current EU and CH proposals: Verifier, holder, and issuer.
Note that roles are tied to interactions, not entities. 
A single entity can assume different roles between transactions or even in a single one. e.g., verifying a credential in order to issue a new one.

%\todo{LG: is it OK if this is very Swiss-related? I suppose that most of it is the same in the EU and elsewhere... }
%\todo{Okay I thought this was related to the previous paragraph. It is true the definitions are very Swiss oriented. I changed some to make them broader. I think we should eventually evolve that so the definitions can be applied elsewhere}

\vspace{1em}

\dirtree{%
.1 Actors. 
.2 Legal Entities. 
.3 Private Individuals. 
.3 Business. 
.3 Government. 
.2 Functional Roles. 
.3 Issuer. 
.3 Holder. 
.3 Verifier. 
}

\paragraph{Legal Entities} define the responsibilities and rights of the different actors in the system.
For Switzerland, these are defined in the \eid Act \cite{BGEID24} and the accompanying ordinance \cite{VEID25}.
Other acts related to digital security will also apply.

\subparagraph{Private Individuals}
are the most common entities using \eid to prove one or more attributes related to their identity.
The Swiss proposal currently only defines a digital identity for individuals.
This makes them \emph{holders} by default. We can envision individuals acting in other \emph{functional roles} even though the technical means to do so may not be available.

\subparagraph{Businesses}
must have a Unique Enterprise Identification Number (UID) to appear as such and be represented officially in the Swiss e-ID system.
In the scope of the Swiss proposal, they are \emph{verifiers}.
It is a natural extension, however, to assume they will eventually act as \emph{holders} and \emph{issuers} too.

\subparagraph{Government}
is the responsible for the system and runs the necessary infrastructure for it, with the exception of the online identity verification using video.
They also run the \emph{registry portal} which allows anybody to sign up and to create entries in the \emph{base registry} for their public keys, as well as entries in the \emph{trust registry}.
The entries in the \emph{trust registry} are verified by the government, and include the full name, address, and reasons to be a verifier or an issuer.

\paragraph{Functional Roles}

The functional roles are also defined in the Swiss \eid Act \cite{BGEID24} and the ordinance \cite{VEID25}.
The EUID \cite{EUDI-ARF} defines the same functional roles, as it is based upon a similar decentralized system like \swiyu.

%\todo{Add a figure here with the functional roles}

\subparagraph{Issuer} creates credentials and distributes them to the \emph{holder}s.
A holder usually starts out with receiving an \eid as their first credential, which can only be issued by the government.
In \swiyu, every issuer can sign up to the base- and trust-registries, but this is not mandatory.
However, if an issuer is not in these registries, the holder will get a notification that they are about to store a credential from a non-official issuer.

\subparagraph{Holder} has an application which stores the \eid and other credentials on an electronic device, most often a smartphone with an Android or an iOS system.
In \swiyu, the holder has full control over their \eid and credentials, and can control at any time which information is presented to whom.

\subparagraph{Verifier} is the role of the entity wanting to receive a proof of one or more fields of a credential from a holder.
The verifier has the legal obligation to ensure only the minimal amount of information necessary for the fulfillment of the task gets requested to the holder.
The Swiss ordinance allows non-registered verifiers to request data, but the holders will be informed that this action is dangerous and not normal \cite[Art. 14]{VEID25}.
For registered verifiers, the Swiss ordinance also contains information about abuse of verification requests.
When a verifier requests information from a holder, the application will show which information the verifier requests, and the holder can decide whether or not to create a proof for these information.

\subsubsection{Trust Goals}
\label{subsub:trust_goals}

\vspace{1em}

\dirtree{%
.1 Trust Goals. 
.2 User Agency. 
.3 Anonymity (Data Minimization). 
.4 Unlinkability. 
.4 Predicate Proofs. 
.3 Autonomy (Decentralization).
.4 Coercion Resistance. 
.4 User-centric Identity. 
.4 Offline Usage. 
.2 Business Interests. 
.3 Fraud prevention. 
.3 Audit Trails. 
.2 Government Requirements. 
.3 Unforgeability of \eid. 
.3 Non-transferability (holder binding). 
.3 Government Oversight / Transparency. 
}

We map each \emph{Actor} type to distinct \emph{Trust Goals}. \emph{Private Individuals} seek \emph{User Agency}, \emph{Businesses} seek trade continuity and improvement, and \emph{Governments} aim to ensure only legitimate institutions can issue identities for specific purposes.
Note that some goals exist prior to any digital identity systems, some are specific to them. Some goals go beyond the identity emission/presentation interactions and cannot be addressed by digital credential technologies. 
They exist nonetheless and should be taken into account when designing the credential layer. We give a more detailed breakdown of these categories.


\paragraph{User Agency} describes the \emph{anonymity} requirements of a user, as well as the \emph{autonomy} that forms the foundation of individual agency in modern society.

\subparagraph{Anonymity} Just as digital payment technologies have increased the fluidity of money, we expect digital identity technologies will do the same for identity. We regroup under \emph{Anonymity} the characteristics of an identity system that enables safe and well-bounded usage of a person's identity. The category is composed of characteristics that limit the risk of an individual's overexposure when using their identity. We favor the term \emph{Anonymity} over \emph{Data Minimization} to retain a focus on individuals rather than on data. We use the definition from \cite{ph10}: "Anonymity of a subject means that the subject is not identifiable within a set of
subjects, the anonymity set."

\textbf{Unlinkability.} Again we use the definition from \cite{ph10}: ''Unlinkability of two or more \emph{Items of Interest} (IOIs, e.g., subjects, messages, actions, ...)
from an attacker’s perspective means that within the system (comprising these and
possibly other items), the attacker cannot sufficiently distinguish whether these IOIs are
related or not.''
We further define \emph{verifier unlinkability} as unlinkability with the attacker being a set of verifiers colluding in the attempt of linking IOIs (sometimes called multi-show unlinkability), and \emph{issuer unlinkability} as unlinkability with the attacker being the IOI issuers, sometimes also called \emph{untraceability}.


\textbf{Predicate Proofs.} By predicate proofs, we mean any technique allowing the holder of a piece of information to make statements about it without revealing it. This can be zero-knowledge if observers cannot infer any knowledge from the statement being proved, or not. A classic example is proving one is over a certain age. Making that statement does not reveal one's birth date but still leaks information about one's age range.


\subparagraph{Autonomy} Identity systems must be resilient to various adverse conditions, here we describe the characteristics of a system allowing a person to use their identity even in the presence of coercive or disruptive powers in the system. \emph{Coercion Resistance} relates to identity infrastructure resisting shifts in political benevolence. For example, can identities be altered, deactivated, or cloned by the issuer after issuance has happened?


\textbf{User-centric Identity} We use user-centric as a category to evaluate the level of control but also the level of liability bestowed on users of a particular identity system. The popular term \emph{Self-Sovereign identity} is one instance of a user-centric identity system in which participants are given the technical and legal means to produce assertions about themselves regardless of their legal status and without relying on non-neutral third parties. We intend to use this category to gauge the self-determination power that participants have but also the risks they are exposed to due to the system's architecture. We also use the category to assess the level of power and technical decentralization of the systems. It should not come at the cost of a loss of control over the identity (identity theft).

\textbf{Offline Usage.} Physical identification documents can be used even without connectivity. We try to capture how much of this property an \eid can retain. We define \emph{offline} as a settings in which parties cannot rely on connectivity to a central server or the global internet, because it is either inexistent or intermittent. We do not consider cases where a device is "never online", whether intentionally or not. Using an \eid in these settings usually mean employing peer-to-peer transport such as Bluetooth, or scanning QRCodes. It also usually requires pre-loading of the \eid and of a certain number of trust elements such as issuers' keys, trust registries, revocation lists, etc.

\paragraph{Business Interests} represent the needs of the business to function, but also how they can comply to the legal requirements given by the laws.

\subparagraph{Fraud prevention} It is notoriously difficult to trust a party's identity over the wire ("on the internet nobody knows you're a dog"). It is mandatory that parties relying on shared data can trust the data to be authentic, unaltered, and belonging (or be entrusted to) the party presenting it. With this category, we evaluate the difficulty for holders to present fictitious data and pass them as authentic.

\subparagraph{Audit Trails} Handling identification data from first-parties comes with legal requirements for businesses. The format and amount of information in these exchanges might help or hinder compliance to local and global regulations. This requires identity systems to be able to operate on various modes of privacy as zero-knowledge exchange come in direct clash with Know Your Customer (KYC) regulation such as Anti-Money Laundering (AML). An analysis of such compromises in identity systems can be found in \cite{ABCD25}.

\paragraph{Government Requirements} come from the goals of the governments, described in the constitution and the applicable laws.

\subparagraph{Unforgeability} A base requirement, not only for \eid, is that it must be (computationally) infeasible to make claims in the name of another party in a way that is indistinguishable from claims made by that party. This is closely related to \emph{Fraud prevention}.

\subparagraph{Non-transferability (holder binding)} If it is hard for malicious actors to forge an identity, the authenticity of shared data might be guaranteed, but the lawful possession of the data by the presenting party is not. The issuers, in our case, the governments, want to ensure that authentic identity data can only be used by authorized people: An \eid should only be usable by the person it represents or a legal guardian. The closest approximation to ensure a particular \eid can only be presented successfully by the authorized person is making sure that an \eid can only be presented from the device it was issued to. This mechanism is usually called \emph{holder binding}.

\subparagraph{Government oversight / transparency} It is sometimes desirable, or even required, for authorities to be able to have insight into how identities are used. Specifically in the AML case already mentioned, identities tied to transactions must retain transparency to the regulator's eyes. This usually comes in direct conflict with \emph{User agency} goals and requires compromises of either the technological or legal form. Auer et al. published a taxonomy of privacy and oversight measures \cite{ABCD25}, in particular, we want to note their use of different "hardness" categories for enforcement techniques: \emph{soft/hard privacy} and \emph{soft/hard oversight}.

\subsubsection{Threat Models}
\label{subsub:threat_models}

To complete the \emph{\rot} branch, we list the broad characteristics that \emph{Threat Models} should consider when modeling the security of electronic identities.
\emph{External Threats} groups threats by entities that are not actors participating in the system in any role or legal status and as such do not have access to private keys to make them legitimate actors.
\emph{Internal Threats} emanate from misbehavior -- intentional or not -- from system participants in any role or legal status. This category models both technical and legal power imbalance.
Finally \emph{Systemic Failures} lists threats to the whole construction, impacting all actors. In particular, we list threats that compromise the cryptographic guarantees of the system: mathematical assumptions, implementation errors, hardware security, as well as the supply chain of all these --- from scientific papers to silicon chips and implementations.

\vspace{1em}

\dirtree{%
.1 Threat Models. 
.2 External Threats. 
.3 Nation State. 
.3 Hacking Group. 
.3 Theft / Device Loss. 
.2 Internal Threats. 
.3 Functional Role Abuse. 
.3 Legal Role Abuse. 
.3 Collusion between Actors. 
.2 Systemic Failures. 
.3 Quantum Computers. 
.3 Cryptographic Soundness. 
.3 Implementation Failure. 
.3 Supply Chain Attacks. 
}

%\todo{LG: Need to go through Fabrice's master thesis and look what is still missing here...}

\paragraph{External Threats}

In this paragraph we consider threats from actors who start with public access
to the system.
This means that if they gain access to any of the secrets in the system, they will
not behave rationally in the sense of the usual owners of these secrets:
While a \emph{holder} will not want to share their secrets with a \emph{verifier},
once an external threat gets hold of these secrets, they will use it even if it adds
damage to the holder.

\subparagraph{Nation States}
\label{subp:nation_state}

We consider threats from \emph{Nation State}s as the most difficult to advert, as they have
the biggest resources available from all threat actors.
From the technical resources, some of these actors can be considered to have a \emph{global network view}
\cite{TorAttack}, which can break anonymity guarantees based on partial network view only.
Another technical resource is the \emph{store now decrypt later}, which allows such an adversary
to store large amounts of traffic data, in the hope to decrypt it once large generic quantum computers
are available.

A nation state also has soft power, which can be used to \emph{coerce} a company into giving
access to its data or operating system \cite{TelegramArrest}\cite{ProtonLogging}.
While \emph{data access} is the most common request done by governments, there have been attempts
to ask a company to modify its operating system to access a device (cite St Bernardino case
\cite{AppleSanBernardino}.
We suppose that such a change could not stay hidden for a very long time, and will have major
repercussions into the trust assumptions given to the companies involved.

A last element is the \emph{modification of law}, to allow retrieval of data which has been
kept hidden from the government beforehand.
In the best case, the changes of law only change the future handling of data (cite
"Vorratsdatenspeicherung"), but sometimes a government will also try to access
data which had guarantees of privacy in the past.

The fact that a nation state can be itself an issuer is handled in \ref{p:internal-threats}.

\subparagraph{Hacking Groups}

We frame hacking groups as technically and financially capable groups using stolen data as a mean to a financial end.
We model them as having less power and capabilities than nation states --- although some might be backed by one.

These groups can use hard power, raw technical capabilities, to attack victims. 
An example is the \emph{denial of service}, in which attackers
overwhelm the capability of the victim's system to handle requests, rendering it unresponsive.

The use of soft power includes \emph{phishing} as the most common attack vector according to \cite{IC3-24}.
Phishing attacks are a form of \emph{social engineering}, where attackers take advantage of human psychology to get victims to act on the behalf of the attackers.
Through the use of emails, SMS, voice calls, or other channels, the attackers get the victims to execute of malicious programs, to send credentials to the attackers, or to commit other actions that will help attackers compromise the victim's system.
It can be targeted towards a specific individual, or rely on targeting a large amount of legitimate users in the hope that at least one (or some) will inadvertently help the attacker.

According to the same report, the most common action from an attacker is \emph{extortion through ransomware}.
Hacker groups \emph{steal data} and request money in exchange for the promise not to divulge the data,
or encrypt the data and give the key in exchange for money.
Victims have to pay without guarantees to receive the key or that greater divulgation will not happen anyway.
Sometimes both methods are combined, and the victim is extorted more than once, which is called \emph{double extortion}
and even has a game-theoretical equilibrium \cite{Meurs24}.

\subparagraph{Theft / Device Loss / Coercion}

We model \emph{theft}, or \emph{loss of a device}, as a threat.
In cases where the device of a legitimate user of the system falls into malicious hands, it must be secured enough to avoid misuse or abuse the victim's identity.

This can also happen through \emph{coercion} of the holder by a third party to use their
credential for a purpose that is unlawful or otherwise harmful to the legitimate user.

\paragraph{Internal Threats}
\label{p:internal-threats}

What can happen if one or more of the official \emph{actors} in the \eid system
misbehave and does not follow the protocol or the legal framework?

\subparagraph{Functional Role Abuse}

Depending on the functional role, different attacks are possible. 

In an \eid system, a \textbf{verifier} can \emph{request too many attributes} 
from the holder, over-identifying them, possibly de-anonymizing them. 
Any overreach produces data records that, in some cases, should not be in possession of the verifier, and are liable to later unwanted disclosure through attacks or mishandling.
Without an \eid system in place, the users will send a copy of their national identity card, which is a juicy target for any criminal finding it \cite{Tea25}.

Verifiers misbehaving by \emph{keeping data} for longer than allowed or agreed expose themselves and holders to the same risks of data exposure. 
In addition, it exposes the holder to future abuse by the verifier.
The Swiss e-ID act defines which type of data can be retained by the system.

An \textbf{issuer} has extended power in an \eid system and has to be trusted 
to fulfill its role in a reliable and trustworthy way.
% For later: Issuers' responsibilities start with the modelization of the credentials they will issue, these should contain only necessary information, avoid unique correlators, have a well-bounded time-to-live. \todo{CH: This begs for an "etc." Maybe we should remove it entirely ?} 
During emission of the credentials, an issuer centralizes a lot of \emph{toxic information}, which must be securely deleted: the link between the signatures and the holders,
the IP address of the holders for requests during a presentation of the certificate, and any information the holder had to present in exchange for the credential.

The issuer must also be trusted to
not \emph{re-issue} a credential to a third party and to ensure they are delivering credentials to the legitimate holder of the information.

\textbf{Holders} have little incentive to break the system, as it is hard for them to do so without raising suspicion and consequences outweigh benefits
more clearly than for legal entities cases.
The likely most frequent misbehavior is \emph{credential sharing}, where a holder lends their credential to another holder.

\subparagraph{Legal Role Abuse}

As described in \ref{subp:nation_state}, any protections on legal aspects should be watched carefully.
Multiple \eid systems in the northern countries rely on the law to protect participants'
data, which is stored with private actors \cite{BankID25}.
These protections are vulnerable to slow regulatory drift, significant \emph{changes of law}, or legal battles and delays.
All these events can drastically alter the rights of governments and/or private companies over the exchanged data.
Individuals can have a hard time following and combating these changes.
%In Switzerland there is a discussion about \emph{extension of the surveillance infrastructure}\cite{DigiGesMassen24}, which will
%significantly weaken the current protection of individuals. \todo{This feels like the beginning of a tangent. Should we cut that ? I think this, and some other of our tangents would make wonderful material for the blog. Although it might be too political for both our organizations}

%Citing Carmela Troncoso from \cite{Troncoso20}
%\begin{quote}
%When you have a secret and you tell it to someone, you trust this person to not reveal it. Privacy by design means that you don’t tell the secret. So you don’t need to trust this person not to reveal it.    \todo{Looking at this individually we might want to give this quote some prime space at the beginning of the next paper's versions. It is quite a good introduction to what we want to do.}
%\end{quote}

A more subtle attack is the \emph{legal loophole} abuse, where certain aspects of the systems
are not defined, be it on purpose, or through negligence.
This can allow governments, or private companies, to collect and abuse
more data than a privacy preserving system should allow.

\subparagraph{Collusion between Actors}

This is a different threat model where actors are not misbehaving in isolation but collaborating together.

If \textbf{multiple verifiers} collude and exchange data about their visitors,
it becomes easier to uniquely identify and correlate their users.
To do this, the verifiers need to \emph{link} the various presentations of the users,
and then consolidate the information each verifier has about the user.
This linkage can happen by \emph{comparing metadata} like IP addresses, access
time, access patterns, or other data given by the user.
For \eid, this can also mean using parts of the certificate to link various presentations:
 the electronic signature, the time of emission, end of validity, revocation number,
or any other data being in common between different presentations of the certificates.
This is happening at large scale when visiting websites, and it's the ad-industry who
does the aggregation of these information to create a user profile \cite{BARW16}.

In the case of collusion between \textbf{one or more verifiers and the issuer},
the available attack surface increases.
As the issuer has knowledge of all the information of the holders, being able to
link any information between the issuer and the verifiers directly gives full
information to all parties.
This could happen if the issuer doesn't follow protocol and stores
\emph{electronic signatures along with the corresponding user identities}.
Together with the information gathered by the verifiers, this gives a more precise
image of what, when, and where the user was using their certificates.

Currently we cannot think of specific threats for \textbf{verifier and holder collusion},
or \textbf{collusion between holders}.

\paragraph{Systemic Failures}

We talk about systemic failures when technical assumptions made when setting up the systems turn out to be false or invalidated (even temporarily).
Some of these assumptions, such as the existence of "powerful enough" quantum computers, have a shrinking shelf life, and systems must be designed accordingly.
Other assumptions rely on the quality of the system's implementation, as well as
the research behind the algorithms used.

Erica Klarreich asks a fundamental question in \emph{which computational universe do we live in?}\cite{KlarUniverse22}
The idea behind this question is the fact that until now, we do not have any mathematical proof of the absolute difficulty
of breaking any of the available cryptographic schemes.
All we can do is compare them to one another in relative terms, and prove statements like: 
''Supposing that factorization of big numbers is hard, RSA is secure''.
And, in effect, factorization of big numbers is hard, unless there is a big enough generic quantum computer.
So the above holds up until the point somebody figures out a an efficient way to factorize large numbers or until the apparition of powerful enough \emph{quantum computers}.
Or if somebody finds a better algorithm to factorize large numbers. 

\subparagraph{Quantum Computers}

Before talking about the effect of quantum computers on cryptographic assumptions, let's make a small detour to understand how we model an important piece of asymmetric cryptography: \emph{one-way functions}, as described in
\ref{sp:public-key-cryptography}.
A one way function $f$ is an operation mapping elements in a set $\mathcal{A}$ to elements in a set $\mathcal{B}$ very efficiently. But, for now, there is no (computationally) easy way, knowing an element $b \in \mathcal{B}$, to compute the corresponding element $a \in \mathcal{A}$ such that $f(a) = b$.
It has been shown theoretically that a large-scale, high-quality, universal
quantum computer \cite[s. 2.1]{TaurusQuantum23} can do this inversion in polynomial time for two of the most used one-way functions: 
\emph{integer factorization}, which is the one-way function of RSA, 
and \emph{discreet logarithm}, which is the one-way function in Elliptic Curve cryptography.
Publicly known quantum computers in 2025 are only \emph{universal}, but lack
the scale and the quality to perform any useful calculation.

Finding the private key, given the public key, has different implications for
the available cryptographic services \ref{pa:cryptographic-services}.
Encryption \ref{sp:encrypting} is the most affected by such capabilities, as the discovery
of the private key allows an attacker to decrypt the message.
In \eid systems, however, vulnerable one-way functions are only used for
signatures \ref{sp:signing}.
If an attacker gets the private key, they will only be able to sign
on the behalf of the holder. This makes all future presentations invalid but does not compromise prior existing records.

\subparagraph{Cryptographic Soundness}

There is a saying in cryptographic circles: \emph{don't roll your own crypto} \cite{SchneierLaw11}.
This is the wisdom gained by many cryptographic protocols which have not been
studied and scrutinized before being used in real products \cite{Vaudenay02}.
In general it is very difficult to create secure cryptographic protocols, and even
the most commonly used like RSA and Elliptic Curves are deemed secure more by the
absence of attacks, than by any mathematical proofs that they are secure.

Most of the cryptographic algorithms used in \eid s like \swiyu rely on the
following assumptions:
\begin{itemize}
    \item Elliptic Curves are a \emph{secure one-way function}
    \item SHA256 is a good representation of the \emph{random oracle model}
\end{itemize}
If these assumptions turn out to be flawed, most of the foundations of \eid and other systems
are broken.

Another aspect of cryptographic soundness is the correctness of proofs for any of the cryptographic
algorithms used in the system.
New algorithms like Longfellow \cite{FS24} and Microsoft's Crescent \cite{FFL25} use complex constructions to optimize the creation of their
Zero-Knowledge Proofs.
It is yet to be shown that these constructions are actually secure, and don't 
\emph{leak information}, or allow the prover to create a \emph{wrong proof}.
Oftentimes, these systems are so complicated, that more than one completely different
team must look at the system, to give some assurance that the construction holds.
Theoretically it should be possible to give a \emph{formal proof} of the constructions,
but so far we don't know of any formal proof of a construction with these
complexities.

There have also been reports of \emph{backdoored algorithms} \cite{DualEC},
where a nation state proposed an algorithm which has been weakened on purpose.
Most of the modern algorithms now avoid \emph{hand-optimized values}, and use
randomly chosen values, or other \emph{nothing-up-my-sleeve number}\cite{Salsa20}.

A final point, which is close to \emph{implementation failures}, is the addition
of \emph{simple} solutions to complex problems.
Such \emph{patches} to existing software often only appear to solve the problem,
when in fact it only hides it badly and doesn't resist to an attack by a serious
attacker.
Unless a solution as been peer-reviewed by established cryptographers, it should
be considered to fall under the ''Don't roll your own crypto'' \cite{SchneierLaw11} caveat.

\subparagraph{Implementation Failures}

Once all the algorithms are written down, they have to be implemented in code.
One of the most common implementation failure leading to security issues is missing
\emph{memory safety}.
If an attacker can get read access to unauthorized memory, or, worse, can overwrite
memory with a \emph{buffer overflow}, they can get access to usually protected
information.
For high-level languages like Java, C\#, or Dot-Net, buffer overflow in the language
is mostly impossible.
However, often they use libraries which are written in low-level languages like
C or C++, where buffer overflows can readily happen.
Other low-level computer languages like \emph{go} or \emph{rust} make these attacks very unlikely
to happen, but these languages are still very rarely used in these cases \cite{HeartbleedXKCD}.

Another common failure is \emph{missing tests} for \emph{edge cases} in the code.
If an attacker can find a critical edge case, it can be used to enter the system.
To avoid this, \emph{external code reviews} are needed, as well as \emph{bug bounty programs},
which allow knowledgeable users to test the system for errors before and after
deployment.

Deployment itself is another critical moment in software development:
\emph{Missing separation} of the debugging, testing, and production code can lead, for
example, to accepting test certificates in production code.
Especially on github, there are many cases where public repositories contain
access tokens, which allow the hackers to enter the system - even from
github itself \cite{GithubPrivate23}.

Less common errors (and harder to exploit too) are \emph{side channel attacks}, which allow the attacker to gain
information about the system by observing it carefully.
The early attacks of this kind used \emph{timing side channels}, later leakages also
involved \emph{electromagnetic side channels}, or \emph{power drawing side channels}.
Depending on the value of the target, these more obscure side channel leakages might 
be enough to break a system.
For the timing side channels, there have been attacks which were even able to be done
over the internet \cite{BB03}.

\paragraph{Supply Chain Attacks}

A very specific implementation failure is the use of unverified or badly tested
software libraries.
Current software projects include hundreds, if not thousands of software libraries.
Most of them are not directly required by the project, but are indirectly included
in the project in one way or another.
Most of today's software languages treat all these libraries equally, so if one of
these libraries has been corrupted by an attacker, and is included in the system,
the whole system can be exposed to this attacker.

However, libraries need to be updated, not only for the sake of getting new capabilities,
but also because updates often fix errors in these libraries.
Keeping track of all the used libraries is a very hard task, and often times it is not
known which library in which version is used in which project \cite{Log4Shell21}.

There is also a potential attack vector with hardware supply chain attack:
A nation state could for example backdoor a hardware security module (HSM), 
and exfiltrate the private key through different means \cite{SchneierBackdoor25}.

\subsection{\sfi}
\label{sub:foundations}

This section lists \emph{existing} constructs such as \emph{cryptographic primitives}, \emph{hardware security}, \emph{data structures}, and \emph{communication protocols} (for a later release) related to digital identity systems. Together, they make up a set of tools that we will consider when building a proposal for the \eid.

\subsubsection{Cryptographic Foundations}
\label{subsub:cryptography}

This section lists \emph{existing} cryptographic building blocks, schemes, and protocols.

\dirtree{%
.1 Cryptographic Foundations. 
.2 Primitives. 
.3 Hashes. 
.3 Public Key Cryptography. 
.3 Symmetric Cryptography. 
.2 Cryptographic Services.
.3 Signing. 
.3 Exchanging keys. 
.3 Encrypting. 
.2 Privacy-first techniques. 
.3 Zero-Knowledge Proofs (ZKP). 
.3 Selective Disclosure. 
.3 Private information retrieval (PIR). 
.2 Post-Quantum Security. 
}

\paragraph{Primitives} Cryptographic primitives form the foundational mathematical operations upon which all higher-level security protocols are built.
They providing the basic computational building blocks for secure communication and data protection.

\subparagraph{Hashes} They are used to create a digital fingerprint, the \emph{hash value}, of a source, the \emph{message}.
The size of the hash value is always the same, regardless of the size of the message.
A \emph{cryptographic hash function} needs to have the following properties:
the output must be \emph{equiprobable} for any input, finding the input of a random output is only possible through enumeration, \emph{pre-image resistance}, given an input/output pair, only enumeration can give a second input with the same output, \emph{second pre-image resistance}, randomly finding two inputs with the same output is only possible with enumeration, \emph{collision resistance}.
Aside from SHA-256 \cite{FIPS180-4} and SHA-3 \cite{FIPS202}, we want to mention Poseidon \cite{GKRRS21}, a purpose-specific hash function designed to be cost-effective when used in zk-SNARKS constructions.

\subparagraph{Public-Key Cryptography} 
\label{sp:public-key-cryptography}
Public key (or asymmetric) cryptography relies on a pair of keys, one \emph{public}, the other \emph{private}, and a one-way function. 
One-way functions are easy to compute in one direction, and hard in the other. 
This allows a party to encrypt a message using the other party's public key. 
Message that can then only be decrypted by someone in possession of the corresponding private key.

\subparagraph{Symmetric Cryptography} Symmetric cryptography relies on parties agreeing on a common secret to protect information. 
Symmetric systems are computationally very efficient and cryptographic systems usually rely on a combination of asymmetric and symmetric cryptographic primitives.

\paragraph{Cryptographic services} 
\label{pa:cryptographic-services}
We describe here the elementary services required for a secure and private \eid. \emph{Encrypting} encompasses symmetric and asymmetric schemes designed to preserve the confidentiality of data during transmission or storage. The distributed, many-actors nature of distributed identity systems makes them well-suited for public-key architectures, which enable secure \emph{key exchanges} between parties without prior shared secret, as well as the publication of verification material for \emph{signatures}.

\subparagraph{Signing} 
\label{sp:signing}
Signing protocols, providing authentication and integrity guarantees, have seen extensive development relevant to digital identity needs. Of particular relevance to this work are signature schemes that allow easy blinding of signatures (CL \cite{CL02}, BLS \cite{BLS01}, BBS+\cite{ASM06}), and signature schemes already widely used, whether in digital identity schemes or not (ECDSA \cite{NIST186-5}, EdDSA \cite{BDLSY11}, Schnorr)

\subparagraph{Exchanging keys}
\label{sp:exchanging-keys}
As mentioned previously, symmetric cryptography is efficient but in a lot of scenarios, establishing a common secret in a safe way is not a trivial matter. This is where asymmetric cryptography helps. Key-exchange protocols such as the one proposed by Diffie and Hellman \cite{DH76} and all the variants it prompted enable two parties to agree on a common key based on---and this is a crucial point---the input of both parties.

\subparagraph{Encrypting} 
\label{sp:encrypting}
Encryption schemes is broadly split between symmetric (AES \cite{NIST2001AES}, ChaCha20-Poly1305 \cite{RFC7539}) and asymmetric (RSA \cite{RSA78}, ElGamal \cite{E85}) systems. It is also quite common to build hybrid systems in which public-key cryptography is used to encrypt a symmetric key which is then used to encrypt communication and benefit from the speed-advantage of symmetric schemes.

\paragraph{Privacy-first techniques.} Under this section we file techniques that can alleviate the tension between privacy (\emph{User agency}) and authorities oversight by providing techniques enabling strong security while limiting the amount of data shared in the process.

\subparagraph{Zero-Knowledge Proofs} Zero-Knowledge proofs are getting an unprecedented amount of attention with their blockchain applications. 
Despite a very different setup, they are also being studied and developed for digital identity applications.
Whereas blockchain applications typically optimize for proof length in priority, to reduce on-chain storage size, our study focuses on provers' run-time first. In the context of digital identity, provers are typically running on resource-constrained hardware (consumer phones). Proof size is also an important factor --- proofs need to be transmitted over the network as fast as possible.  

In particular, we discuss the following:

\begin{itemize}
    \item commitment schemes (Pedersen \cite{PD91}, Polynomial commitment schemes)
    \item models and components for proof systems and protocols (sigma protocols \cite{SP90}, Sumcheck \cite{LCF92}, Ligero \cite{AHIV22})
    \item zk-SNARKS (Groth16 \cite{G16}, PLONK \cite{GWC90}, Pinocchio \cite{PHGR13}), zk-STARKs (zk-STARK \cite{BCG18}), and their cousins (Bulletproof \cite{BBD18})
\end{itemize}

\subparagraph{Selective Disclosure} By selective disclosure we mean any mechanism that allows holders of signed information to reveal only specific portions of it while preserving the verifier's ability to validate the integrity and provenance of shared data. This can be achieved through characteristics of the signature scheme (CL, BBS+ \cite{ASM06}) or through the construction of the credential itself (IETF SD-JWT VC \cite{SDJWT}, ISO mDoc/mDL \cite{ISO18013-5_2021}).
The latter do not embed the claims in the credentials, but use salted hashes signed by the issuer so that individual pieces of information revealed by the holders can be linked back to the signed credential by verifiers.

\subparagraph{Private Information Retrieval} Most credential systems in production today use simple status lists to indicate the revocation or suspension status of issued credentials (IETF Token Status List \cite{TSL}, W3C Bitstring Status List \cite{W3C-status-lists}), In the context of proving one's credentials non-revocation, we discuss Private Information Retrieval schemes such as \cite{MW22} that could allow provers to fetch information concerning their credentials revocation status without revealing any information to issuers. In Swiyu \cite{Swiyu} it might also be a way to access the Trust Registries without revealing information to the entity operating the infrastructure.

\paragraph{Post-quantum security} As we already mentioned, the question of post-quantum security of the whole digital identity system is of notable importance to our discussion.
It is particularly relevant in the fact that many of the techniques that are convenient to build anonymous credentials (such as BBS+ signatures \cite{ASM06}), are not quantum-resistant and usually not part of existing hardware security modules' feature set. If regulation requires \emph{non-transferability} to use hardware binding, any non-hardware-supported scheme might require years for proper standardization and industry roll-out --- years during which the need for quantum-resistance might become more pressing, or simply necessary.


\subsubsection{Hardware Security}

Here we make a distinction between Secure Enclave (SE) or Trusted Platform Modules (TPM), typically required for individual's computing devices, and Hardware Security Modules (HSMs), specialized hardware that is already legally enforced for providers of security services such as (Qualified) Trust Service Providers in the EU.

The immutable nature of these hardware security features creates long-term dependencies that must be carefully considered. Any change to the required feature set represents high time and money costs, provided that the incentive for change exists at all. This, combined with the looming post-quantum security requirement, is a big source of discussion regarding the trade-offs in \eid systems.

\dirtree{%
.1 Hardware Security. 
.2 Secure Element. 
.3 Apple Secure Enclave.
.3 Android Trusty. 
.2 Hardware Security Module. 
}

\paragraph{Secure Element} We are primarily concerned with the secure element of mobile devices, since it is the most likely solution to achieve holder binding on consumer phones.

\subparagraph{Apple Secure Enclave}

Apple Secure Enclave \cite{apple-secure-enclaves} is a coprocessor implementation found across Apple's device ecosystem, including iPhones, iPads, Macs, and other Apple silicon-based products. Among other features, it offers dedicated hardware for encryption using AES, RSA as well as elliptic curve cryptography.
Furthermore it allows to sign messages using RSA and elliptic curve cryptography, secure storage, key management, and key attestation capabilities.

\subparagraph{Android Trusty}

On most Android devices, the secure OS (Trusty \cite{trusty}) relies on ARM TrustZone \cite{arm-trustzone}. The latter provides hardware-enforced separation between secure and non-secure worlds within ARM-based processors.
For our purpose, these environments are sensibly similar to Apple's, offering encryption, signing, and secure storage operations.


\subparagraph{Key attestations} Key attestations are useful in proving to third parties that a key has been well-generated and is actually hardware-bound. Sometimes they also prove that the device they were generated on is not jailbroken. There is a major difference here between Apple and Google's implementation of key attestations. Apple's attestations require Apple itself to verify the key information, whereas Google's can be verified by third parties. This could have an impact on the use of secure elements and key attestation in the scope of public institutions' \eid projects.

\paragraph{Hardware Security Module} Hardware security modules are physically separate hardware appliances. They are designed for efficient key storage, and cryptographic operations (in particular, signatures). They are also designed to be tamper-resistant and provide physical security. This kind of hardware is usually mandated for actors allowed to issue and sign certificates in a public-key infrastructure, or in banks. The same kind of requirements are enforced for actors allowed to issue high level of assurance attestations in the EU ARF \cite{EUDI-ARF}, such as the \emph{Personal Identification Data (PID)}.


\subsection{\rnf}
\label{sub:regulatory}

The last part of our taxonomy describes the 
\emph{\rnf} considered in this paper.


\subsubsection{Legal Texts} 

For this paper we only consider the EU and the Swiss legal texts.
In our taxonomy, we consider the Swiss \emph{Constitution} to be
at the same level as the EU \emph{Treaties}.
They build the base for all legal activities, have a very general
language, and are rarely modified.
Following the constitution, Switzerland has \emph{Law}s and the EU
creates \emph{Regulation}s.
These legal documents contain still very abstract language, but
are used as the basis for the actual applied documents:
\emph{Ordinance}s in Switzerland, and 
\emph{Implementation Regulation}s in the EU.
They are applied and contain more concrete language
indicating what MAY, SHOULD, and MUST be implemented.

\vspace{1cm}

\dirtree{%
.1 Legal Texts. 
.2 Constitution Treaties. 
.2 Law Regulation. 
.3 Switzerland's e-ID Act. 
.3 EU and the EIDAS/EUDI. 
.2 Ordinances Directives. 
.3 Switzerland's Ordinance Proposal. 
.3 EU's Implementation Regulations. 
}

\paragraph{Constitution Treaties}

The EU \eid is based on the article 26 of the \emph{Treaty on the Functioning of the European Union}
\cite{EUTreaties}{Art.26}, which defines the internal market of the EU:
\begin{quote}
    1.   The Union shall adopt measures with the aim of establishing or ensuring the functioning of the internal market, in accordance with the relevant provisions of the Treaties.
\end{quote}
In addition, the \emph{Charter of fundamental rights} defines the privacy and personal data protection:
\begin{quote}
    Article 7 - Respect for private and family life - Everyone has the right to respect for his or her private and family life, home and communications.
    
    Article 8 - Protection of personal data - 1.   Everyone has the right to the protection of personal data concerning him or her.
    - 2.   Such data must be processed fairly for specified purposes and on the basis of the consent of the person concerned or some other legitimate basis laid down by law. Everyone has the right of access to data which has been collected concerning him or her, and the right to have it rectified. - 3.   Compliance with these rules shall be subject to control by an independent authority.
\end{quote}

The proposed Swiss \eid act is based on the article 81, of the \emph{Swiss Constitution} \cite{SwissConstitution}, \emph{Public Construction Works}.

\paragraph{Law / Regulation}

The EU published a first regulation in 2015 called EIDAS\cite{EIDAS}.
It was not mandatory, but allowed all member states to start working on an
interoperable \eid.
In Summer 2024, the EU published the updated EIDAS regulation 
officially called EUDI\cite{EUDI}, unofficially called EIDASv2.
The EUDI makes it mandatory for the EU countries to implement an EU-wide accepted
\eid solution within 24 months.

For Switzerland, a first \eid act has been refused by the Swiss people in 2021\cite{CHeID21}.
One of the most voiced critiques in 2021 was the emission and data handling by
private actors.
For this reason, a new \eid act has been worked out, based on government-run services,
and the Swiss people will vote on this new act in September 2025\cite{SwiyuAct25}.

\paragraph{Ordinances and Directives}

In the EU five implementation regulations have been proposed\cite{EUDIReg24}:
four describing the technical implementation, which will be
updated regularly, and a fifth, which is a \emph{Notification to the commission}.
In addition to these acts, the EU has defined the \emph{Architecture and Reference Framework
}(ARF)\cite{EUARF25}.
The ARF describes the technical details of the implementation and is under active
discussion on their github repository.

In June 2025, the Swiss government published a proposed ordinance for the
\eid act\cite{SwiyuOrdinance25}.
It describes in more details how the system should work, what data needs to be stored,
and how the different actors work together.
Building on its experience with the CoVid App \cite{SwisscovidGithub}, Switzerland is also
developing the \swiyu app in a github repository, and is open to discussions
and PRs from a broader public \cite{SwiyuGithub}.
